\documentclass[11pt]{article}

\usepackage{amsmath}
\usepackage{amsfonts}
\usepackage[margin=1in]{geometry}
\usepackage{enumitem}
\usepackage{graphicx}
\usepackage[colorlinks]{hyperref}
\usepackage{longtable}

\usepackage{helvet}
\renewcommand{\familydefault}{\sfdefault}

\setlength{\parindent}{0in}

\def\tightlist{}
\def\toprule{}
\def\bottomrule{}

\begin{document}
{\LARGE Homework 7 (Due October
28th)}\label{homework-7-due-october-28th}

{\Large 1. Multipole Expansion of a Single Point
Charge}\label{multipole-expansion-of-a-single-point-charge}

For this problem, consider a single point charge \(+q\).

\begin{enumerate}
\def\labelenumi{\arabic{enumi}.}
\tightlist
\item
  Place the charge at the origin, write down the electric potential at a
  location \(\mathbf{r} = \langle x,y,z \rangle\) from the origin.
\item
  Move the charge to a short distance away from the origin on the
  \(z\)-axis, \(\mathbf{r}' = \langle 0,0,d\rangle\). Write down the
  electric potential at a location
  \(\mathbf{r} = \langle x,y,z \rangle\) from the origin.
\item
  Assume the location of interest in Part 2 is far from the charge
  (\(r>>d\)). Expand your result in Part 2 keeping only the two leading
  order terms. Interpret these terms in light of the multipole
  expansion. \emph{Hint: It might help to rewrite your result in
  spherical coordinates.}
\item
  How do you resolve that your answer to Part 1 only contains a monopole
  term where your answer to Part 3 contains additional terms? Explain
  your reasoning.
\end{enumerate}

{\Large 2. A Curious Sphere of
Charge}\label{a-curious-sphere-of-charge}

In this problem, we ask you to plot a few functions. You have plotted
quite a bit using Jupyter, so we expect that you will use a Jupyter
notebook of your own design to do your plotting now.

Consider a sphere of radius \(R\) that has a volume charge density
inside the sphere given by:

\[\rho(r,\theta) = \mu r \sin\left(\dfrac{3\theta}{2}\right)\]

where \(\mu\) is known constant and \(\theta\) is the usual polar angle
in spherical coordinates.

\begin{enumerate}
\def\labelenumi{\arabic{enumi}.}
\tightlist
\item
  Plot \(\rho(r,\theta)/r\) in units of \(\mu\) as a function of
  \(\theta\). Where does this charge live in space? Note that
  \(\rho \propto r\).
\item
  Calculate the total charge, \(Q\), on the sphere.
\item
  Calculate the dipole moment, \(\mathbf{p}\), of the sphere.
\item
  Use your results from Parts 2 and 3 to find \(V(r,\theta)\) when you
  are far from the sphere (\(r>>R\)). Discuss how your results make
  sense with the plot in Part 1.
\item
  The function
  \(\sin\left(\frac{3\theta}{2}\right) \approx \frac{1}{\sqrt{2}} + \frac{3}{2\sqrt{2}}\cos\theta\),
  which would suggest that the volume charge density can be written as
  \(\rho(r,\theta) \approx \frac{\mu r}{\sqrt{2}} + \frac{3\mu r}{2\sqrt{2}} \cos \theta\).
  So this will look like a superposition of a spherically symmetric
  density and a density proportional to \(\cos \theta\). Plot
  \(\rho(r,\theta)/r\) in units of \(\mu\) as a function of \(\theta\)
  in this approximation.
\item
  How does your plot in Part 1 compare to your plot in Part 5? Is this a
  good approximation to the original charge density? What does this
  imply about our approximation of \(V\) compared to the exact \(V\)?
\end{enumerate}

{\Large 3. Atomic hydrogen and the polarization
model}\label{atomic-hydrogen-and-the-polarization-model}

Griffiths Table 4.1 gives an experimental value for
\(\alpha/4\pi\varepsilon_0\) for atomic hydrogen. (Read his caption
carefully for units!)

\begin{enumerate}
\def\labelenumi{\arabic{enumi}.}
\tightlist
\item
  The ``atomic polarizibility'', \(\alpha\) is defined by
  \(\mathbf{p}=\alpha\mathbf{E}\). Study Griffiths' Example 4.1, which
  tells you how to estimate the atomic polarizability, summarize the
  example in your own words.
\item
  Following the example and using it with this experimental value for
  \(\alpha/4\pi\varepsilon_0\) for atomic hydrogen, estimate the atomic
  radius of hydrogen. How well did you do, compared, say, with the Bohr
  radius?
\item
  After summarizing the example, tell us what physical assumption
  (simplification!) Griffiths is making about the physical distribution
  of negative charge inside an atom? Is that realistic?
\item
  Now suppose you have a single hydrogen atom inside a charged
  parallel-plate capacitor, with plate spacing 1 mm, and voltage 100 V.
  Determine the ``separation distance'' \(d\) (as defined in that same
  Example 4.1 problem) of the electron cloud and the proton nucleus.
  What fraction of the atomic radius of part 2 is this? (You should
  conclude that 100 V across a 1mm gap capacitor is unlikely to ionize a
  hydrogen atom, do you agree?)
\item
  Use your calculations to roughly estimate what voltage (and thus, what
  E-field) would ionize this single hydrogen atom. (We'd say if you can
  pull the electron cloud one full atomic radius away, it's breaking
  down! )
\end{enumerate}

{\Large 4. Polarized sphere of
charge}\label{polarized-sphere-of-charge}

Consider a dielectric sphere of radius \(a\) that has a polarization
that is directed radially outward from the center of the sphere,
\(\mathbf{P} = P_0\mathbf{r}\).

\begin{enumerate}
\def\labelenumi{\arabic{enumi}.}
\tightlist
\item
  Determine the bound charges at the surface, \(\sigma_B\), and in the
  volume of the sphere, \(\rho_B\).
\item
  Find the electric field everywhere.
\item
  Sketch the electric field lines inside and outside the sphere. What
  does your sketch say about the electric field at the boundary of the
  sphere? Does this make sense to you? Why or why not?
\end{enumerate}

{\Large 5. The bar electret}\label{the-bar-electret}

A curious little device that is the electrical analog of the bar magnet
is the bar electret. It is a short cylinder with a radius of \(b\) and a
length \(l\) that carries a uniform polarization \(\mathbf{P}\) along
its axis. In this problem, you will sketch the electric field produced
by the bar electret for several scenarios.

\begin{enumerate}
\def\labelenumi{\arabic{enumi}.}
\tightlist
\item
  Find the bound charge everywhere in or on the bar electret.
\item
  Sketch and describe the electric field produced by the bar electret if
  its length is much greater than its radius (long and skinny,
  \(L>>b\)).
\item
  Sketch and describe the electric field produced by the bar electret if
  its length is much smaller than its radius (short and fat, \(L<<b\)).
\item
  Sketch and describe the electric field produced by the bar electret if
  its length is roughly equal to its radius (\(L\approx b\)).
\end{enumerate}

{\Large 6. Charge conservation}\label{charge-conservation}

When a neutral dielectric is polarized, no new charges are created or
destroyed, so the total charge must still be zero. The charge density on
the surface is given by:

\[\sigma_B = \mathbf{P}\cdot\hat{n}\]

The charge density in the bulk is given by:

\[\rho_B = -\nabla \cdot \mathbf{P}\]

Using these definitions, show that the total charge for any neutral
dielectric with a polarization \(\mathbf{P}\) is zero.
\end{document}
