\documentclass[11pt]{article}

\usepackage{amsmath}
\usepackage{amsfonts}
\usepackage[margin=1in]{geometry}
\usepackage{enumitem}
\usepackage{graphicx}
\usepackage[colorlinks]{hyperref}
\usepackage{longtable}

\usepackage{helvet}
\renewcommand{\familydefault}{\sfdefault}

\setlength{\parindent}{0in}

\def\tightlist{}
\def\toprule{}
\def\bottomrule{}

\begin{document}
{\LARGE Homework 4 (Due September
30th)}\label{homework-4-due-september-30th}

Homework 4 emphasizes another alternative method to direct integration
for solving the electric field problem by reducing the vector problem to
a scalar one by using electric potential. In addition, it introduces the
electric potential energy concept. This homework emphasizes 2.3 and 2.4,
but Sec. 2.2 (Gauss' Law) continues to be very important.

\paragraph{1. Energy of a point charge
distribution}\label{energy-of-a-point-charge-distribution}

When studying crystal structures (e.g., in condensed matter physics), it
is sometimes convenient to model those structures as rectangular grids
of charged ions, this problem offers a starting point for such a model.

Imagine a small square (side \(a\)) with four point charges \(+q\), one
on each corner.

\begin{enumerate}
\def\labelenumi{\arabic{enumi}.}
\tightlist
\item
  Calculate the total stored energy of this system (i.e.~the amount of
  work required to assemble it).
\item
  Calculate how much work it takes to ``neutralize'' these charges by
  bringing in one more point charge (\(-4q\)) from far away and placing
  it right at the center of this square.
\end{enumerate}

\paragraph{2. Screened Coulomb
Potential}\label{screened-coulomb-potential}

In a
\href{./homework3.html\#connecting-potential-electric-field-and-charge}{previous
problem}, you worked out the electric field and charge distribution for
a point charge using the electric potential. In this problem, you will
gain some additional practice doing this for the
\href{https://en.wikipedia.org/wiki/Electric-field_screening}{screened
Coulomb potential}.

Consider the ``screened Coulomb potential'' of a point charge \(q\) that
arises, for example, in plasma physics:

\[V(r) = \dfrac{q}{4\pi\varepsilon_0} \dfrac{e^{-r/\lambda}}{r}\]

where \(\lambda\) is a constant (called
\href{https://en.wikipedia.org/wiki/Debye_length}{the screening
length}).

\begin{enumerate}
\def\labelenumi{\arabic{enumi}.}
\tightlist
\item
  Determine the electric field \(\mathbf{E}(\mathbf{r})\) associated
  with this potential.
\item
  Find the charge distribution \(\rho(\mathbf{r})\) that produces this
  potential. (Think carefully about what happens at the origin!)
\item
  Sketch this function \(\rho(\mathbf{r})\) in a manner that clearly
  describes its characteristics (i.e., what's the best way of
  representing this three-dimensional charge distribution? Use it, and
  explain what you are plotting.)
\item
  Show, by explicit calculation over \(\rho(\mathbf{r})\) that the net
  charge represented by this distribution is zero. (\emph{If you don't
  get zero, think again about what happens at \(r = 0\).}).
\item
  Verify this result using the integral form of Gauss' Law (i.e.,
  integrate your electric flux over a \emph{very large} spherical
  surface.)
\end{enumerate}

\paragraph{3. Finding voltage from a charge
distribution}\label{finding-voltage-from-a-charge-distribution}

We have found a number of ways of relating \(\rho\), \(\mathbf{E}\), and
\(V\). In this problem, you will use \(\rho\) to find \(V\) through the
method of direct integration (i.e., using the integral expression for
\(V\)).

\begin{enumerate}
\def\labelenumi{\arabic{enumi}.}
\tightlist
\item
  Find a formula for the electrostatic potential \(V(z)\) everywhere
  along the symmetry-axis of a charged ring (radius \(a\), centered on
  the \(z\)-axis, with uniform linear charge density \(\lambda\) around
  the ring). Please use the method of direct integration to do this, and
  set your reference point to be \(V(\infty)=0\).\\
\item
  Sketch \(V(z)\), how does \(V(z)\) behave as \(z \rightarrow \infty\)?
  (Don't just say it goes to zero. How does it go to zero?) Does your
  answer make physical sense to you? Explain briefly.
\item
  Use your result from part 1 for \(V(0,0,z)\) to find \(z\)-component
  of the electric field anywhere along the \(z\)-axis?
\item
  What is the voltage at the origin? What is the electric field at the
  origin? Do these results from \(V\) and \(\mathbf{E}\) at the origin
  make physical sense to you, and are they consistent with each other?
  Briefly explain.
\end{enumerate}

\paragraph{4. Capacitors, metals, and
continuity}\label{capacitors-metals-and-continuity}

We have discovered that there is a curious result when looking at the
electric field across a boundary. It is discontinuous! by an amount that
is consistently the same expression (\(\sigma/\varepsilon_0\)). However,
we have also found that the electric potential across the boundary is
continuous.

Consider a very large set of metal capacitor plates separated by a small
distance \(d\). Each plate is charged with an equal amount of charge,
but the left plate is negative. So, the left plate carries \(-\sigma\)
and the right plate carries \(+\sigma\).

\begin{enumerate}
\def\labelenumi{\arabic{enumi}.}
\tightlist
\item
  Find the electric field everywhere in space (consider that the plates
  extend to infinity to ignore any edge effects).
\item
  Plot the component of the electric field along a line that runs to the
  plates.
\item
  Find the electric potential everywhere in space. You are free to set
  where \(V=0\), but be careful when doing so as the distribution of
  charge extends to infinity!
\item
  Plot the electric potential along the same line as in Part 2.
\item
  What do you notice about the graphs you sketched in Parts 2 and 4?
\item
  BONUS: Consider that the plates have some depth to them (as with any
  real physical plate). Let them have a width \(w\). Using what you
  remember or know about metals, what happens to your graphs in Parts 2
  and 4? (For example, we will learn soon that metals in electrostatics
  are equipotential surfaces.)
\end{enumerate}

\paragraph{5. Surface charge and boundary
conditions}\label{surface-charge-and-boundary-conditions}

It might seem to you that the results that the electric field is
discontinuous by an amount \(\sigma/\varepsilon_0\) isn't really a big
deal. There's probably a question about how useful this result is. We
will come back to this particularly when we get to fields in matter, and
suffice it to say, it will help us a lot there. To get a flavor of what
is coming, this problem will discuss this discontinuity in a familiar
context.

\begin{enumerate}
\def\labelenumi{\arabic{enumi}.}
\tightlist
\item
  Consider a cylindrical metal rod (radius \(r\), length \(L\)) with a
  constant charge density \(\sigma\) distributed across its outer
  surface (as we will learn that is the only place the charge can be).
  Using Gauss' Law (far from the ends of the rod; assume it's long and
  skinny), determine the electric field inside and outside the rod.
\item
  Take the difference between the electric fields you determined in Part
  1 (technically, the perpendicular component) across the outer surface
  of the metal rod to show you recover the the result that the all the
  charge lives on the surface.
\item
  Consider a similarly cylindrical plastic rod with a constant charge
  density \(\rho\) distributed over its entire volume. Again, using
  Gauss' Law (far from the ends of the rod; assume it's long and
  skinny), determine the electric field inside and outside the rod.
\item
  Again, take the difference between the electric fields you determined
  in Part 3 across the outer surface of the plastic rod. What do you
  find? Does your result make physical sense?
\end{enumerate}

\paragraph{6. An energy conundrum}\label{an-energy-conundrum}

There's a bit of a conundrum that occurs when we begin to compare our
two different descriptions of energy associated with electrostatic
systems. In this problem, you will compare these descriptions and
develop an argument that resolves the conundrum.

Consider two point charges (\(q_1\) and \(q_2\)) that are brought to be
a distance \(r\) apart. You can locate them anywhere to develop this
argument, but for the sake clarity, let's put them on the \(x\)-axis
straddling the origin (i.e., one at \(r/2\) and the other at \(-r/2\)).

\begin{enumerate}
\def\labelenumi{\arabic{enumi}.}
\tightlist
\item
  First, compute the work done to bring the charge configuration
  together. Recall that it costs nothing (i.e., there's no work done) to
  bring the first charge to it's location. Does this expression look
  familiar?
\item
  Now, construct the integral expression for the total energy associated
  with the charge configuration using the integral formalism:
  \(\frac{\varepsilon_0}{2} \int E^2 d\tau\). Remember that the electric
  field in this integral expression is due to the field from both
  charges: \(\mathbf{E} = \mathbf{E}_1 + \mathbf{E}_2\). \emph{Do not
  try to integrate it.}
\item
  Your integral expression can be expanded out to three terms: in
  principle, you can integrate one of the terms, but not the other two.
  Which two can't you integrate and why not?
\item
  What is the physical significance of the two un-integrable terms? What
  must the integrable term be?
\end{enumerate}
\end{document}
